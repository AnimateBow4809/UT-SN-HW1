%%%%%%%%%%%%%%%%%%%%%%%%%%%%%%%%%%%%%%%%%
% Cleese Assignment (For Students)
% LaTeX Template
% Version 2.0 (27/5/2018)
%
% This template originates from:
% http://www.LaTeXTemplates.com
%
% Author:
% Vel (vel@LaTeXTemplates.com)
%
% License:
% CC BY-NC-SA 3.0 (http://creativecommons.org/licenses/by-nc-sa/3.0/)
% 
%%%%%%%%%%%%%%%%%%%%%%%%%%%%%%%%%%%%%%%%%

%----------------------------------------------------------------------------------------
%	PACKAGES AND OTHER DOCUMENT CONFIGURATIONS
%----------------------------------------------------------------------------------------

\documentclass[11pt]{article}
\usepackage{amssymb}
\usepackage{float}  
\usepackage{subcaption}
\input{structure.tex} % Include the file specifying the document structure and custom commands

%----------------------------------------------------------------------------------------
%	ASSIGNMENT INFORMATION
%----------------------------------------------------------------------------------------

% Required
\newcommand{\assignmentQuestionName}{Question} % The word to be used as a prefix to question numbers; example alternatives: Problem, Exercise
\newcommand{\assignmentClass}{Social Networks} % Course/class
\newcommand{\assignmentTitle}{Assignment\ \#1} % Assignment title or name
\newcommand{\assignmentAuthorName}{Ali Dashtbozorg 810104302} % Student name

% Optional (comment lines to remove)
\newcommand{\assignmentClassInstructor}{Dr. Masoud Asadpour 15:00am} % Intructor name/time/description
\newcommand{\assignmentDueDate}{Monday,\ December\ 15,\ 2025} % Due date

%----------------------------------------------------------------------------------------

\begin{document}

%----------------------------------------------------------------------------------------
%	TITLE PAGE
%----------------------------------------------------------------------------------------

\maketitle % Print the title page

\thispagestyle{empty} % Suppress headers and footers on the title page

\newpage

%----------------------------------------------------------------------------------------
%	QUESTION 1
%----------------------------------------------------------------------------------------

\begin{question}

\questiontext{The Watts-Strogatz Model}
 

\begin{subquestion}{Consider the small-world model of Watts and Strogatz with rewiring probability p . The
		model starts with a regular ring lattice where each node is connected to its ⟨k⟩ / 2 neigh-
		bors on either side for a total degree of ⟨k⟩ .1 Show that when p = 0 , the overall clustering
		coefficient of this graph is given by: \\
		\begin{equation}
			C(0) = \frac{{3(k - 2)}}{{4(k - 1)}}
	\end{equation}}
	\answer{
		We know that the global clustering coefficient is given by equation \ref{formula:gc}:
		\begin{equation}
			\label{formula:gc}
			C = \frac{3 \times \text{number of triangles}}{\text{number of connected triples of vertices}}
		\end{equation} 
		we know that for the WS model the number of connected triples of vertices is just N times the number of triplets each node gives which is given by equation \ref{formula:trip}:
		\begin{equation}
			\label{formula:trip}
N\left( {\begin{array}{*{20}{c}}
		{\langle k\rangle } \\ 
		2 
\end{array}} \right)
		\end{equation}
		for the number of triangles for each node we need to consider 2 cases:
		\begin{enumerate}
			\item  where the node has triangles with only one side of its connections so the number of triangles 
			is given by \ref{formula:halftri}
			\begin{equation}
				\label{formula:halftri}
				\left( {\begin{array}{*{20}{c}}
						{\frac{{\langle k\rangle }}{2}} \\ 
						2 
				\end{array}} \right) + \left( {\begin{array}{*{20}{c}}
						{\frac{{\langle k\rangle }}{2}} \\ 
						2 
				\end{array}} \right)
			\end{equation}
			\item  where each node has triangles with one node on one side and the other on another side the number of triangles os given by \ref{formula:fulltri}
			\begin{equation}
				\label{formula:fulltri}				
				0 + 1 + 2 + 3 + 4 + ... + \frac{{\langle k\rangle }}{2} - 1 = \frac{{(1 + \frac{{\langle k\rangle }}{2} - 1)(\frac{{\langle k\rangle }}{2} - 1)}}{2} = \left( {\begin{array}{*{20}{c}}
						{\frac{{\langle k\rangle }}{2}} \\ 
						2 
				\end{array}} \right)
			\end{equation}
		\end{enumerate}
		When we sum the results from equation \ref{formula:halftri},\ref{formula:fulltri} we are given 3 times the number of triangles so for the actual number of triangles we need to divide by 3 putting all of this together with equation \ref{formula:trip} we get equation \ref{formula:gc2}
		\begin{equation}
			\label{formula:gc2}			
			\begin{gathered}
				C(0) = \frac{{3(\frac{{3N\left( {\begin{array}{*{20}{c}}
										{\frac{{\langle k\rangle }}{2}} \\ 
										2 
								\end{array}} \right)}}{3})}}{{N\left( {\begin{array}{*{20}{c}}
								{\langle k\rangle } \\ 
								2 
						\end{array}} \right)}} = \frac{{3\frac{{(\frac{{\langle k\rangle }}{2})(\frac{{\langle k\rangle }}{2} - 1)}}{2}}}{{\frac{{\langle k\rangle (\langle k\rangle  - 1)}}{2}}} = \frac{{3(\frac{{\langle k\rangle }}{2})(\frac{{\langle k\rangle }}{2} - 1)}}{{\langle k\rangle (\langle k\rangle  - 1)}} = \frac{{\frac{3}{4}(\langle k\rangle )(\langle k\rangle  - 2)}}{{\langle k\rangle (\langle k\rangle  - 1)}} \hfill \\
				\;\;\;\;\;\;\;\;\;\;\;\;\;\;\;\;\;\;\;\;\;\;\;\;\;\;\;\;\;\;\;\;\;\;\;\;\;\;\;\;\;\;\;\;\;\;\;\;\;\;\;\;\;\;\;\;\; = \frac{{3(\langle k\rangle  - 2)}}{{4(\langle k\rangle  - 1)}} \hfill \\ 
			\end{gathered} 
		\end{equation}
	}
\end{subquestion}

\begin{subquestion}{Show that when $p > 0$ , the overall clustering coefficient is given by:
	\begin{equation}
C(p) \approx \frac{{3(k - 2)}}{{4(k - 1)}}{(1 - p)^3}
		\end{equation}}
	\answer{In this question we refer back to equation \ref{formula:gc} since the degree of nodes isn't dependent on  $p$ the number of connected triples will not change with changing $p$  the only thing that will change is the number of triangles for that case we can make an induction that the number of triangles we have at $p=0$ called $t(0)$ will survive in $p=p_1$ with probability that all 3 triangles edges survive thus $t(p)=t(0)(1-p) ^3$ and for the global clustering coefficient we reach equation \ref{formula:cof}:
	\begin{equation}
		\label{formula:cof}
		C(p) \approx C(0){(1 - p)^3} = \frac{{3(k - 2)}}{{4(k - 1)}}{(1 - p)^3}
		\end{equation}    }
\end{subquestion}
\end{question}

\begin{question}
	content...
\end{question}

\end{document}