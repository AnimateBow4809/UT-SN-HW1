%%%%%%%%%%%%%%%%%%%%%%%%%%%%%%%%%%%%%%%%%
% Cleese Assignment (For Students)
% LaTeX Template
% Version 2.0 (27/5/2018)
%
% This template originates from:
% http://www.LaTeXTemplates.com
%
% Author:
% Vel (vel@LaTeXTemplates.com)
%
% License:
% CC BY-NC-SA 3.0 (http://creativecommons.org/licenses/by-nc-sa/3.0/)
% 
%%%%%%%%%%%%%%%%%%%%%%%%%%%%%%%%%%%%%%%%%

%----------------------------------------------------------------------------------------
%	PACKAGES AND OTHER DOCUMENT CONFIGURATIONS
%----------------------------------------------------------------------------------------

\documentclass[11pt]{article}
\usepackage{amssymb}
\usepackage{float}  
\usepackage{subcaption}
\input{structure.tex} % Include the file specifying the document structure and custom commands

%----------------------------------------------------------------------------------------
%	ASSIGNMENT INFORMATION
%----------------------------------------------------------------------------------------

% Required
\newcommand{\assignmentQuestionName}{Question} % The word to be used as a prefix to question numbers; example alternatives: Problem, Exercise
\newcommand{\assignmentClass}{Social Networks} % Course/class
\newcommand{\assignmentTitle}{Assignment\ \#1} % Assignment title or name
\newcommand{\assignmentAuthorName}{Ali Dashtbozorg 810104302} % Student name

% Optional (comment lines to remove)
\newcommand{\assignmentClassInstructor}{Dr. Masoud Asadpour 15:00am} % Intructor name/time/description
\newcommand{\assignmentDueDate}{Monday,\ December\ 15,\ 2025} % Due date

%----------------------------------------------------------------------------------------

\begin{document}

%----------------------------------------------------------------------------------------
%	TITLE PAGE
%----------------------------------------------------------------------------------------

\maketitle % Print the title page

\thispagestyle{empty} % Suppress headers and footers on the title page

\newpage

%----------------------------------------------------------------------------------------
%	QUESTION 1
%----------------------------------------------------------------------------------------

\begin{question}

\questiontext{The Watts-Strogatz Model}
 

\begin{subquestion}{Consider the small-world model of Watts and Strogatz with rewiring probability p . The
		model starts with a regular ring lattice where each node is connected to its ⟨k⟩ / 2 neigh-
		bors on either side for a total degree of ⟨k⟩ .1 Show that when p = 0 , the overall clustering
		coefficient of this graph is given by: \\
		\begin{equation}
			C(0) = \frac{{3(k - 2)}}{{4(k - 1)}}
	\end{equation}}
	\answer{
		We know that the global clustering coefficient is given by equation \ref{formula:gc}:
		\begin{equation}
			\label{formula:gc}
			C = \frac{3 \times \text{number of triangles}}{\text{number of connected triples of vertices}}
		\end{equation} 
		we know that for the WS model the number of connected triples of vertices is just N times the number of triplets each node gives which is given by equation \ref{formula:trip}:
		\begin{equation}
			\label{formula:trip}
N\left( {\begin{array}{*{20}{c}}
		{\langle k\rangle } \\ 
		2 
\end{array}} \right)
		\end{equation}
		for the number of triangles for each node we need to consider 2 cases:
		\begin{enumerate}
			\item  where the node has triangles with only one side of its connections so the number of triangles 
			is given by \ref{formula:halftri}
			\begin{equation}
				\label{formula:halftri}
				\left( {\begin{array}{*{20}{c}}
						{\frac{{\langle k\rangle }}{2}} \\ 
						2 
				\end{array}} \right) + \left( {\begin{array}{*{20}{c}}
						{\frac{{\langle k\rangle }}{2}} \\ 
						2 
				\end{array}} \right)
			\end{equation}
			\item  where each node has triangles with one node on one side and the other on another side the number of triangles os given by \ref{formula:fulltri}
			\begin{equation}
				\label{formula:fulltri}				
				0 + 1 + 2 + 3 + 4 + ... + \frac{{\langle k\rangle }}{2} - 1 = \frac{{(1 + \frac{{\langle k\rangle }}{2} - 1)(\frac{{\langle k\rangle }}{2} - 1)}}{2} = \left( {\begin{array}{*{20}{c}}
						{\frac{{\langle k\rangle }}{2}} \\ 
						2 
				\end{array}} \right)
			\end{equation}
		\end{enumerate}
		When we sum the results from equation \ref{formula:halftri},\ref{formula:fulltri} we are given 3 times the number of triangles so for the actual number of triangles we need to divide by 3 putting all of this together with equation \ref{formula:trip} we get equation \ref{formula:gc2}
		\begin{equation}
			\label{formula:gc2}			
			\begin{gathered}
				C(0) = \frac{{3(\frac{{3N\left( {\begin{array}{*{20}{c}}
										{\frac{{\langle k\rangle }}{2}} \\ 
										2 
								\end{array}} \right)}}{3})}}{{N\left( {\begin{array}{*{20}{c}}
								{\langle k\rangle } \\ 
								2 
						\end{array}} \right)}} = \frac{{3\frac{{(\frac{{\langle k\rangle }}{2})(\frac{{\langle k\rangle }}{2} - 1)}}{2}}}{{\frac{{\langle k\rangle (\langle k\rangle  - 1)}}{2}}} = \frac{{3(\frac{{\langle k\rangle }}{2})(\frac{{\langle k\rangle }}{2} - 1)}}{{\langle k\rangle (\langle k\rangle  - 1)}} = \frac{{\frac{3}{4}(\langle k\rangle )(\langle k\rangle  - 2)}}{{\langle k\rangle (\langle k\rangle  - 1)}} \hfill \\
				\;\;\;\;\;\;\;\;\;\;\;\;\;\;\;\;\;\;\;\;\;\;\;\;\;\;\;\;\;\;\;\;\;\;\;\;\;\;\;\;\;\;\;\;\;\;\;\;\;\;\;\;\;\;\;\;\; = \frac{{3(\langle k\rangle  - 2)}}{{4(\langle k\rangle  - 1)}} \hfill \\ 
			\end{gathered} 
		\end{equation}
	}
\end{subquestion}

\begin{subquestion}{Show that when $p > 0$ , the overall clustering coefficient is given by:
	\begin{equation}
C(p) \approx \frac{{3(k - 2)}}{{4(k - 1)}}{(1 - p)^3}
		\end{equation}}
	\answer{In this question we refer back to equation \ref{formula:gc} since the degree of nodes isn't dependent on  $p$ the number of connected triples will not change with changing $p$  the only thing that will change is the number of triangles for that case we can make an induction that the number of triangles we have at $p=0$ called $t(0)$ will survive in $p=p_1$ with probability that all 3 triangles edges survive thus $t(p)=t(0)(1-p) ^3$ and for the global clustering coefficient we reach equation \ref{formula:cof}:
	\begin{equation}
		\label{formula:cof}
		C(p) \approx C(0){(1 - p)^3} = \frac{{3(k - 2)}}{{4(k - 1)}}{(1 - p)^3}
		\end{equation}    }
\end{subquestion}
\end{question}

\begin{question}
\questiontext{Snobbish Network}
\begin{subquestion}{For a very large, snobbish network where $p >> q $, determine the minimal values for p and
		q (each in terms of N ) required to ensure global connectivity.}
		\answer{
			Since $p >> q $ we have to first connect each community individually and than make atleast 1 connection 
			between the 2 communities for the first part we know from lecture notes that $p > \frac{{\ln N}}{N}$ is sufficient condition for connectivity in each of the communities for there to be atleast 1 connection between the 2 communities  we follow equation \ref{formula:connectivity}
			\begin{equation}
			\label{formula:connectivity}
			\begin{gathered}
				possible\;edges = {N^2} \hfill \\
				p(No\;Edges) = {(1 - q)^{{N^2}}} = {[{(1 - q)^{\frac{1}{q}}}]^{q{N^2}}}\xrightarrow{{p \gg q}} \approx {[{e^{ - 1}}]^{q{N^2}}} = {e^{ - q{N^2}}} \hfill \\
				q{N^2} \geqslant \ln N \to q \geqslant \frac{{\ln N}}{{{N^2}}},p \geqslant \frac{{\ln N}}{N} \hfill \\ 
			\end{gathered} 
			\end{equation}
		}
\end{subquestion}
\begin{subquestion}{What is the expected scaling of the average shortest path length $⟨d_{same}⟩$ between
		two nodes of the same color?}
	\answer{ from the lecture notes we know that equation \ref{formula:davg} holds:
		\begin{equation}
			\label{formula:davg}
			\langle {d_{same}}\rangle  = \frac{{\ln N}}{{\ln \langle k\rangle }} = \frac{{\ln N}}{{\ln Np}} = \frac{{\ln N}}{{\ln N\frac{{\ln N}}{N}}} = \frac{{\ln N}}{{\ln \ln N}}
		\end{equation}
	}
\end{subquestion}

\begin{subquestion}{What is the expected scaling of the average shortest path length $⟨d_{diff} ⟩$ between
		two nodes of different colors?}
	\answer{
		from equation \ref{formula:davg} we know the average distances between 2 nodes of the same color so for the distance between nodes of different colors we first need to get to bridge nodes and from there go to the destination node the total distance is given by equation \ref{formula:ddiff}
		\begin{equation}
			\label{formula:ddiff}			
			\langle {d_{diff}}\rangle  = \langle {d_{same}}\rangle  + 1 + \langle {d_{same}}\rangle  = 2\langle {d_{same}}\rangle  + 1 = \frac{{2\ln N + \ln \ln N}}{{\ln \ln N}} = O(\frac{{\ln N}}{{\ln \ln N}})
		\end{equation}
	}
\end{subquestion}

\begin{subquestion}{Based on your answers, does this network model exhibit the small-world property?1
		Justify your answer by explaining how this model does (or does not) satisfy the two
		key characteristics of small-world networks (path length and clustering).}
	\answer{
		The average distance between nodes scales logarithmically so it has the short path property of small world networks but since it is also a random network it does not have a high clustering coefficient but by definition and lecture notes small world networks don't need to have a high clustering coefficient(unless we are looking for ultra small world property)  so this network is a small world network given that it is connected
	}
\end{subquestion}
\end{question}

\begin{question}
\questiontext{Small World Phenomena in Random Networks}\\
The code for generating the graphs and calculating the shortest path is provided in Helpers section and uses igraph package
\begin{subquestion}{Network Construction and Simulation}
\answer{
		in figure \ref{fig:q2_initial_plot} we have visualized the results for average shortest path which correspond to our theory where 1d lattice have significantly higher shortest path compared to the rest and random networks which have small world property have lower shortest path compared to the rest
}
\begin{figure}[H]
	\centering
	\includegraphics[width=0.85\textwidth,keepaspectratio]
	{imgs/q2_initial_plot.png}
	\caption{linear plot and log-log plot scaling of average path length ⟨d⟩ vs network size N for Lattices and Random
		Networks..}
	\label{fig:q2_initial_plot}
\end{figure}
\end{subquestion}
\begin{subquestion}{Scaling Analysis}
	\answer{
		in figure \ref{fig:q2_second_plot} we  computed the slope derived from our data and it almost is perfect when compared to our theory the only reason theory differs from our data is because when we are talking about scale of the shortest path we don't include coefficient behind it so actually for a 2d lattice it would be $\langle d\rangle  \sim a{N^{\frac{1}{2}}}$ if we were to include $a$ than our theory and simulations would align perfectly
		and as shown on the graph our simulated exponents match the theoretical dimensionality
	}
	\begin{figure}[H]
		\centering
		\includegraphics[width=0.85\textwidth,keepaspectratio]
		{imgs/q2_second_plot.png}
		\caption{comparing simulation results with the theoretical predictions shown in the lecture slide.}
		\label{fig:q2_second_plot}
	\end{figure}
\end{subquestion}
\begin{subquestion}{Explain why the Random Network follows a logarithmic scaling while lattices follow a polynomial scaling. What structural feature of the Random Network (often referred to as “shortcuts”) is responsible for this drastic reduction in path length?}
\answer{
in a regular lattice nodes are connected only to nearby neighbors so reaching distant nodes requires traversing space step by step causing the typical shortest path length to grow as a power of the system size $N$ (specifically $\ell \sim N^{1/d}$ in $d$ dimensions) In contrast a random network contains long range connections often called shortcuts that link nodes which are far apart in the underlying structure. These shortcuts allow the number of reachable nodes to grow exponentially with the number of steps, so the entire network can be reached in a number of steps that scales logarithmically with $N$ ($\langle d\rangle \sim \log N$). The presence of these shortcuts is therefore the key structural feature responsible for the drastic reduction in average path length in random networks.
}
\end{subquestion}
\end{question}

\end{document}