%%%%%%%%%%%%%%%%%%%%%%%%%%%%%%%%%%%%%%%%%
% Cleese Assignment (For Students)
% LaTeX Template
% Version 2.0 (27/5/2018)
%
% This template originates from:
% http://www.LaTeXTemplates.com
%
% Author:
% Vel (vel@LaTeXTemplates.com)
%
% License:
% CC BY-NC-SA 3.0 (http://creativecommons.org/licenses/by-nc-sa/3.0/)
% 
%%%%%%%%%%%%%%%%%%%%%%%%%%%%%%%%%%%%%%%%%

%----------------------------------------------------------------------------------------
%	PACKAGES AND OTHER DOCUMENT CONFIGURATIONS
%----------------------------------------------------------------------------------------

\documentclass[11pt]{article}
\usepackage{amssymb}
\usepackage{float}  
\usepackage{subcaption}
\input{structure.tex} % Include the file specifying the document structure and custom commands

%----------------------------------------------------------------------------------------
%	ASSIGNMENT INFORMATION
%----------------------------------------------------------------------------------------

% Required
\newcommand{\assignmentQuestionName}{Question} % The word to be used as a prefix to question numbers; example alternatives: Problem, Exercise
\newcommand{\assignmentClass}{Social Networks} % Course/class
\newcommand{\assignmentTitle}{Assignment\ \#1} % Assignment title or name
\newcommand{\assignmentAuthorName}{Ali Dashtbozorg 810104302} % Student name

% Optional (comment lines to remove)
\newcommand{\assignmentClassInstructor}{Dr. Masoud Asadpour 15:00am} % Intructor name/time/description
\newcommand{\assignmentDueDate}{Monday,\ December\ 15,\ 2025} % Due date

%----------------------------------------------------------------------------------------

\begin{document}

%----------------------------------------------------------------------------------------
%	TITLE PAGE
%----------------------------------------------------------------------------------------

\maketitle % Print the title page

\thispagestyle{empty} % Suppress headers and footers on the title page

\newpage

%----------------------------------------------------------------------------------------
%	QUESTION 1
%----------------------------------------------------------------------------------------

\begin{question}

\questiontext{The Watts-Strogatz Model}
 

\begin{subquestion}{Consider the small-world model of Watts and Strogatz with rewiring probability p . The
		model starts with a regular ring lattice where each node is connected to its ⟨k⟩ / 2 neigh-
		bors on either side for a total degree of ⟨k⟩ .1 Show that when p = 0 , the overall clustering
		coefficient of this graph is given by: \\
		\begin{equation}
			C(0) = \frac{{3(k - 2)}}{{4(k - 1)}}
	\end{equation}}
	\answer{
		We know that the global clustering coefficient is given by equation \ref{formula:gc}:
		\begin{equation}
			\label{formula:gc}
			C = \frac{3 \times \text{number of triangles}}{\text{number of connected triples of vertices}}
		\end{equation} 
		we know that for the WS model the number of connected triples of vertices is just N times the number of triplets each node gives which is given by equation \ref{formula:trip}:
		\begin{equation}
			\label{formula:trip}
N\left( {\begin{array}{*{20}{c}}
		{\langle k\rangle } \\ 
		2 
\end{array}} \right)
		\end{equation}
		}
\end{subquestion}

\begin{subquestion}{What is the action?}
	\answer{In this question we define the action as choosing which one of the offers to make to the customer   }
\end{subquestion}

\begin{subquestion}{How should the ’reward’ be defined? (Hint: It must represent the net economic value of an action.
		Consider both the value of keeping a customer and the cost of the specific offer.)}
		
\answer{
	We now model the reward as a mixed Bernoulli–Normal process to better reflect realistic variability in the economic outcome of a successful retention. 
	
	For each action $i$, let $p_i$ denote the probability that the customer is retained (success), and $c_i$ the cost of the offer.  
	If the customer is retained, the reward follows a Normal distribution centered around the net gain $(300 - c_i)$, while failure (customer churn) yields a small fixed loss.  
	
	Formally,
	\begin{equation}
		r_i =
		\begin{cases}
			\mathcal{N}(\mu_i, \sigma_i^2), & \text{with probability } p_i \text{ (success / retained)}, \\[6pt]
			v_i, & \text{with probability } 1 - p_i \text{ (failure / churn)},
		\end{cases}
		  \label{eq:one}
	\end{equation}
	where typically $\mu_i = 300 - c_i$, $\sigma_i^2$ represents variability in customer value, and $v_i$ is a small negative penalty stated in the question as $-5$.
	
	The expected reward for action $i$ is then
	\begin{equation}
		\mathbb{E}[r_i] = p_i \, \mu_i + (1 - p_i) \, v_i.
	\end{equation}
	
	This formulation corresponds to our \texttt{CustomBernoulliEnvironment}, where the Bernoulli component models customer retention probability and the Normal component captures the economic uncertainty of a successful retention.
}
\end{subquestion}

\begin{subquestion}{What is the overall goal that the agent is trying to achieve during its 1,000 actions?}
	\answer{
		In this task, the agent's goal is to maximize the cumulative expected reward (total profit).  
		Let $r_i$ denote the reward at the $i$th iteration. Then the objective can be written as
		\begin{equation}
			\text{Maximize} \quad \sum_{i=1}^{1000} r_i
		\end{equation}
	}
\end{subquestion}

\end{question}

\end{document}